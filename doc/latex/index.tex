Functions for decoding analog and digital data. The boxing library has functions for coding (\hyperlink{index_boxing}{boxing}) and decoding (\hyperlink{index_unboxing}{unboxing}) digital and analog data.

The \hyperlink{group__boxer}{Boxer} (\hyperlink{boxer_8h_source}{boxing/boxer.h}) is the top level API for boxing data. It takes a byte array as input and converts it to raw-\/images. The layout of the raw-\/image and the coding of the digital data is defined by the \hyperlink{group__config}{Configuration}.

The \hyperlink{group__unboxer}{Unboxer} (\hyperlink{unboxer_8h_source}{boxing/unboxer.h}) takes sampled input images, and decodes in two steps: extract and decode. The extract step locates the frame within the image, decodes the metadata in the bottom border of the frame, then tracks the pixels within the frame. The tracked pixels are then quantified for digital data. The decode step is different for analog and digital data. Digital decode is applying the codec defined by the \hyperlink{group__config}{Configuration}. Analoge decode is applying the LUT created from the frame top border calibration bar.\hypertarget{index_Definitions}{}\subsection{Definitions}\label{index_Definitions}
\hypertarget{index_sampled-image}{}\subsubsection{sampled-\/image}\label{index_sampled-image}
2 dimensional digitized version of image stored on analog storage medium. Must be sampled with higher resoultion than original resolution used when writing the image. \hypertarget{index_raw-image}{}\subsubsection{raw-\/image}\label{index_raw-image}
Two dimensional digital image to be written on analog storage medium. The image represents a rendered 2D barcode image with a frame and a data container. The resoultion of the raw image per printed pixel is from1 to 8 bits. \hypertarget{index_metadata}{}\subsubsection{metadata}\label{index_metadata}
Generic information stored in the border of the frame of a raw image. Examples can be frame number and checksums. \hypertarget{index_boxing}{}\subsubsection{boxing}\label{index_boxing}
Coding analog and digital data into raw images. \hypertarget{index_unboxing}{}\subsubsection{unboxing}\label{index_unboxing}
Decoding sampled images and restoring the original content written to the raw image. \hypertarget{index_boxing-format}{}\subsubsection{boxing-\/format}\label{index_boxing-format}
Parameters describing the raw image geometry and the methods used for coding the digital data into the frame. \hypertarget{index_extract-step}{}\subsubsection{extract-\/step}\label{index_extract-step}
First step in unboxing a sampled-\/image. The process consists of locating the frame within the image, then decoding the metadata in the bottom border of the frame, then tracking the pixels within the frame. The tracked pixels are then quantified for digital data. \hypertarget{index_decode-step}{}\subsubsection{decode-\/step}\label{index_decode-step}
Second step in unboxing a sampled-\/image. The step is different for analog and digital data. Digital decode is applying the codec defined by the boxing\_\-format. Analoge decode is applying the LUT created from the frame top border calibration bar.\hypertarget{index_Samples}{}\subsection{Sample Applications}\label{index_Samples}
\hypertarget{index_Boxer}{}\subsubsection{Boxer}\label{index_Boxer}
Command line application for coding digital data, see tests$\backslash$boxer$\backslash$main.c\hypertarget{index_Unboxer}{}\subsubsection{Unboxer}\label{index_Unboxer}
Command line application for decoding digital data, see tests$\backslash$unboxer$\backslash$main.c 